\documentclass[11pt]{article}
\usepackage{amsmath}
\usepackage{amsfonts,amssymb,verbatim}
\usepackage{amsthm}
\usepackage{mathrsfs}
\usepackage{graphicx}
\usepackage{colortbl,hhline}
\usepackage{fullpage}
\usepackage{calc}
\usepackage[francais]{babel}
\usepackage{fancyhdr}
\usepackage{dsfont}
\usepackage[latin1]{inputenc}
\usepackage{xcolor}
\usepackage{xcolor,rotating,epic,eepic}
\usepackage{tikz-qtree}
\usetikzlibrary{matrix}

\usepackage{fancyhdr}
\usepackage{xcolor,rotating,epic,eepic}
\usepackage{tikz}
\usepackage[babel=true,kerning=true]{microtype}

\usetikzlibrary{%
  arrows,%
  calc,%
  shapes.geometric,%
  shapes.misc,%
  shapes.symbols,%
  shapes.arrows,%
  automata,%
  through,%
  positioning,%
  scopes,%
  decorations.shapes,%
  decorations.text,%
  decorations.pathmorphing,%
  shadows}

\begin{document}

\begin{figure}[h]
   \centering
\begin{tikzpicture}[>=stealth,sloped]

    \matrix (tree) [%
      matrix of nodes,
      minimum size=0.5cm,
      column sep=0.5cm,
      row sep=1.7cm,
    ]
    {  
    
    & & &{\begin{minipage}[t]{0.1\textwidth}\begin{center}O2\end{center} \end{minipage}} &{\begin{minipage}[t]{0.2\textwidth}\begin{center}Alimentation\end{center} \end{minipage}}\\
    &{\begin{minipage}[t]{0.1\textwidth}\begin{center}\textsc{sang}\end{center} \end{minipage}} &{\begin{minipage}[t]{0.1\textwidth}\begin{center}\textsc{qi}\end{center} \end{minipage}} \\
    {\begin{minipage}[t]{0.1\textwidth}\begin{center}TRS\end{center} \end{minipage}} &{\begin{minipage}[t]{0.1\textwidth}\begin{center}\textsc{c\oe{}ur}\end{center} \end{minipage}} &{\begin{minipage}[t]{0.1\textwidth}\begin{center}\textsc{poumon}\end{center} \end{minipage}}\\
    {\begin{minipage}[t]{0.2\textwidth}\begin{center}Diaphragme\end{center} \end{minipage}} & & & & & {\begin{minipage}[t]{0.2\textwidth}\begin{center}\end{center} \end{minipage}}\\
    {\begin{minipage}[t]{0.1\textwidth}\begin{center}TRM\end{center} \end{minipage}} & &{\begin{minipage}[t]{0.1\textwidth}\begin{center}\textsc{rate}\end{center} \end{minipage}} & &{\begin{minipage}[t]{0.1\textwidth}\begin{center}\textsc{estomac}\end{center} \end{minipage}}\\
    & & & &{\begin{minipage}[t]{0.1\textwidth}\begin{center}\textsc{intestin g�le}\end{center} \end{minipage}}\\
    {\begin{minipage}[t]{0.2\textwidth}\begin{center}Ligne passant par l'ombilic\end{center} \end{minipage}} & & & & & {\begin{minipage}[t]{0.2\textwidth}\begin{center}\end{center} \end{minipage}}\\
    {\begin{minipage}[t]{0.1\textwidth}\begin{center}TRI\end{center} \end{minipage}} & &{\begin{minipage}[t]{0.1\textwidth}\begin{center}\textsc{rein}\end{center} \end{minipage}} & &{\begin{minipage}[t]{0.1\textwidth}\begin{center}\textsc{gros intestin vessie}\end{center} \end{minipage}}\\ 
    & & & &{\begin{minipage}[t]{0.1\textwidth}\begin{center}�limination\end{center} \end{minipage}}\\
        };
   
   \draw[-] (tree-4-1) -- (tree-4-6) node [midway,above]{}; %{$P$};
   
   \draw[-] (tree-7-1) -- (tree-7-6) node [midway,above]{}; %{$P$};
   
   \draw[->] (tree-1-4) -- (tree-5-5) node [midway,above]{}; %{$P$};
   \draw[->] (tree-1-5) -- (tree-5-5) node [midway,above]{}; %{$P$};
   
   \draw[->] (tree-5-5) -- (tree-6-5) node [midway,above]{}; %{$P$};
   \draw[->] (tree-6-5) -- (tree-8-5) node [midway,above]{}; %{$P$};
   \draw[->] (tree-8-5) -- (tree-9-5) node [midway,above]{}; %{$P$};
  
    
\end{tikzpicture}
\caption{\textbf{\label {}Shema du \textsc{triple r�chauffeur}}} 
\end{figure}



\end{document}
    
    
    
    
    
    
    
    
    
    
    
    
    
    