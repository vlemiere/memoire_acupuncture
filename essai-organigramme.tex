\documentclass[11pt]{article}
\usepackage{amsmath}
\usepackage{amsfonts,amssymb,verbatim}
\usepackage{amsthm}
\usepackage{mathrsfs}
\usepackage{graphicx}
\usepackage{colortbl,hhline}
\usepackage{fullpage}
\usepackage{calc}
\usepackage[francais]{babel}
\usepackage{fancyhdr}
\usepackage{dsfont}
\usepackage[latin1]{inputenc}
\usepackage{xcolor}
\usepackage{xcolor,rotating,epic,eepic}
\usepackage{tikz-qtree}
\usetikzlibrary{matrix}

\usepackage{fancyhdr}
\usepackage{xcolor,rotating,epic,eepic}
\usepackage{tikz}
\usepackage[babel=true,kerning=true]{microtype}

\usetikzlibrary{%
  arrows,%
  calc,%
  shapes.geometric,%
  shapes.misc,%
  shapes.symbols,%
  shapes.arrows,%
  automata,%
  through,%
  positioning,%
  scopes,%
  decorations.shapes,%
  decorations.text,%
  decorations.pathmorphing,%
  shadows}

\begin{document}

\begin{figure}[h]
   \centering
\begin{tikzpicture}[>=stealth,sloped]

    \matrix (tree) [%
      matrix of nodes,
      minimum size=0.25cm,
      column sep=0.25cm,
      row sep=1.5cm,
    ]
    {                                                                                                    
                                                                                           
1&2 &3 &\fbox{\begin{minipage}[t]{0.1\textwidth}\begin{center}Vide de \textsc{rein}\end{center} \end{minipage}}&5 &6\\       
1&\fbox{\begin{minipage}[t]{0.1\textwidth}\begin{center}Vide de \textsc{qi} du \textsc{rein}\end{center} \end{minipage}} 
&3&\fbox{\begin{minipage}[t]{0.1\textwidth}\begin{center}Vide de \textsc{yang} de \textsc{rein}\end{center} \end{minipage}} 
&5&\fbox{\begin{minipage}[t]{0.1\textwidth}\begin{center}Vide de \textsc{yin} de \textsc{rein}\end{center} \end{minipage}}\\  
\fbox{\begin{minipage}[t]{0.1\textwidth}\begin{center}Vide de \textsc{qi} du \textsc{poumon}\end{center} \end{minipage}}&2&3&4&5&6\\       
\fbox{\begin{minipage}[t]{0.1\textwidth}\begin{center}Diminution de la PO2\end{center} \end{minipage}}
&\fbox{\begin{minipage}[t]{0.1\textwidth}\begin{center}Stase de \textsc{sang}\end{center} \end{minipage}}
&\fbox{\begin{minipage}[t]{0.1\textwidth}\begin{center}Vide de \textsc{yang} de \textsc{rate}\end{center} \end{minipage}}
&\fbox{\begin{minipage}[t]{0.1\textwidth}\begin{center}Absence d'augmentation du d�bit et de la fr�quence cardiaque\end{center} \end{minipage}}
&\fbox{\begin{minipage}[t]{0.1\textwidth}\begin{center}Absence d'augmentation de la vol�mie\end{center} \end{minipage}}
&\fbox{\begin{minipage}[t]{0.1\textwidth}\begin{center}Augmentation de la chaleur\end{center} \end{minipage}}\\
1&2&\fbox{\begin{minipage}[t]{0.1\textwidth}\begin{center}Vide de \textsc{sang}\end{center} \end{minipage}}
&\fbox{\begin{minipage}[t]{0.1\textwidth}\begin{center}Vide de \textsc{liquides organiques}\end{center} \end{minipage}}&\fbox{\begin{minipage}[t]{0.1\textwidth}\begin{center}Hypoperfusion placentaire\end{center} \end{minipage}}&6\\
    };
   
\draw[->] (tree-1-4) -- (tree-2-2) node [midway,above]{}; %{$P$};
\draw[->] (tree-1-4) -- (tree-2-4) node [midway,above]{}; %{$P$};     
\draw[->] (tree-1-4) -- (tree-2-6) node [midway,above]{}; %{$P$};
    
\draw[->] (tree-2-2) -- (tree-4-1) node [midway,below]{}; %{$(1-p)$};
\draw[->] (tree-2-2) -- (tree-4-2) node [midway,below]{}; %{$(1-p)$};
\draw[->] (tree-2-4) -- (tree-4-3) node [midway,below]{}; %{$(1-p)$};
\draw[->] (tree-2-4) -- (tree-4-4) node [midway,below]{}; %{$(1-p)$};
\draw[->] (tree-2-6) -- (tree-4-5) node [midway,below]{}; %{$(1-p)$};
\draw[->] (tree-2-6) -- (tree-4-6) node [midway,below]{}; %{$(1-p)$};

\draw[->] (tree-2-6) -- (tree-5-3) node [midway,below]{}; %{$(1-p)$};
\draw[->] (tree-2-6) -- (tree-5-4) node [midway,below]{}; %{$(1-p)$};

\draw[->] (tree-2-4) -- (tree-5-3) node [midway,below]{}; %{$(1-p)$};

\draw[->] (tree-4-3) -- (tree-2-4) node [midway,below]{}; %{$(1-p)$};

\draw[->] (tree-3-1) -- (tree-4-1) node [midway,below]{}; %{$(1-p)$};
\draw[->] (tree-3-1) -- (tree-4-2) node [midway,below]{}; %{$(1-p)$};

\draw[->] (tree-4-1) -- (tree-5-3) node [midway,below]{}; %{$(1-p)$};
\draw[->] (tree-4-2) -- (tree-5-3) node [midway,below]{}; %{$(1-p)$};
\draw[->] (tree-4-3) -- (tree-5-3) node [midway,below]{}; %{$(1-p)$};
\draw[->] (tree-4-3) -- (tree-5-4) node [midway,below]{}; %{$(1-p)$};
\draw[->] (tree-4-4) -- (tree-5-5) node [midway,below]{}; %{$(1-p)$};
\draw[->] (tree-4-5) -- (tree-5-5) node [midway,below]{}; %{$(1-p)$};



    
\end{tikzpicture}
\caption{\textbf{\label {}Relation entre vide de \textsc{rein} et vide de \textsc{sang}}} 
\end{figure}
\end{document}